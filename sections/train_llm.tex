\chapter{Training LLMs: How to generate the generator}

\section{Multi-GPU environments}
Training is a resource-intensive endeavor. A model that only takes a single GPU to run inference on may take 10 times that many to train if, for nothing else, to parallelize your work and speed things up so you aren't waiting for a thousand years for it to finish training.  

\subsection{Setting up} 
It should be pointed out up front that while multi-GPU environments are powerful, they are also expensive. For the rest of us, setting up a virtual machine (VM) in Google's Compute Engine is one of the easiest methods. 
\paragraph{Google Virtual Machine} One of the easiest ways to create a multi-GPU environment is to set up a VM on Google's cloud. 
\begin{enumerate}
	\item Create a Google Cloud Project (GCP).
		\begin{itemize}
			\item Set up billing
			\item Download the gcloud CLI. 
		\end{itemize}
	\item After setting up your account, GCP sets your GPU quotas to 0. Quotas are used to manage your costs. You need to increase to 2 or more, since we plan to use multiple GPUs.  
	\item Init by ``\texttt{gcloud init}''
	\item 
\end{enumerate}

\subsection{Libraries}

\paragraph{DeepSpeed:} DeepSpeed is an optimization library for distributed deep learning. DeepSpeed is powered by Microsoft and implements various enhancements for speed in training and inference, like handling extremely long or multiple inputs in different modalities, quantization, caching weights and inputs, and, probably the hottest topic right now, scaling up to thousands of GPUs.  

To install,
\begin{enumerate}
	\item Install PyTorch
	\item \texttt{pip install deepspeed}
\end{enumerate}

\paragraph{Accelerate:} From HuggingFace, Accelerate is made to help abstract the code for parallelizing and scaling to multiple GPUs away from you so that you can focus on the training and inference side. 

To install,
\begin{enumerate}
	\item \texttt{pip install accelerate}
\end{enumerate}

\paragraph{PyTorch FSDP}

Install PyTorch with distributed support (CUDA version must match drivers):
\begin{lstlisting}[language=Python]
pip install torch torchvision torchaudio

# (Optional) For speedups
pip install torchmetrics accelerate
\end{lstlisting}

Ensure passwordless SSH between nodes if you're on a bare-metal or HPC cluster.

Create \texttt{train\_fsdp.py}:

\begin{lstlisting}[language=Python]
import torch
import torch.nn as nn
import torch.optim as optim
import torch.distributed as dist
from torch.distributed.fsdp import FullyShardedDataParallel as FSDP
from torch.distributed.fsdp.wrap import transformer_auto_wrap_policy

# --- Simple Transformer block ---
class ToyBlock(nn.Module):
    def __init__(self):
        super().__init__()
        self.fc1 = nn.Linear(1024, 4096)
        self.act = nn.ReLU()
        self.fc2 = nn.Linear(4096, 1024)

    def forward(self, x):
        return self.fc2(self.act(self.fc1(x)))

class ToyModel(nn.Module):
    def __init__(self, depth=6):
        super().__init__()
        self.layers = nn.Sequential(*[ToyBlock() for _ in range(depth)])

    def forward(self, x):
        return self.layers(x)

def main():
    dist.init_process_group("nccl")
    torch.cuda.set_device(dist.get_rank() % torch.cuda.device_count())

    model = ToyModel().cuda()

    # Auto-wrap large layers with FSDP
    auto_wrap_policy = transformer_auto_wrap_policy
    model = FSDP(model, auto_wrap_policy=auto_wrap_policy)

    optimizer = optim.AdamW(model.parameters(), lr=1e-4)

    for step in range(20):
        x = torch.randn(8, 1024).cuda()
        y = model(x).mean()
        y.backward()
        optimizer.step()
        optimizer.zero_grad()
        if dist.get_rank() == 0:
            print(f"Step {step} done.")

    dist.destroy_process_group()

if __name__ == "__main__":
    main()
\end{lstlisting}

If your node has 4 GPUs:

\texttt{torchrun --nproc\_per\_node=4 train\_fsdp.py}
\begin{itemize}
	\item 4 processes (one per GPU).
	\item NCCL backend handles GPU communication.
\end{itemize}

Let's try running multiple nodes

\begin{itemize}
	\item Node 0: 10.0.0.1
	\item Node 1: 10.0.0.2
	\item 4 GPUs per node
	\item Total world size = 8 (2 nodes $\times$ 4 GPUs)
\end{itemize}

\begin{lstlisting}[language=Python]
**On Node 0 (rank 0):**

torchrun --nnodes=2 --nproc_per_node=4 \
         --node_rank=0 \
         --master_addr=10.0.0.1 \
         --master_port=29500 \
         train_fsdp.py

**On Node 1 (rank 1):**

torchrun --nnodes=2 --nproc_per_node=4 \
         --node_rank=1 \
         --master_addr=10.0.0.1 \
         --master_port=29500 \
         train_fsdp.py
\end{lstlisting}
\begin{itemize}
	\item \texttt{--nnodes=2}: total number of nodes.
	\item \texttt{--nproc\_per\_node=4}: GPUs per node.
	\item \texttt{--node\_rank}: each node's unique index.
	\item \texttt{--master\_addr}: IP/hostname of rank 0 node.
	\item \texttt{--master\_port}: open port for coordination.
\end{itemize}

\begin{itemize}
	\item Activation checkpointing: Saves memory by discarding intermediate activations during forward pass and recomputing them during backward pass.
	\item Mixed precision: Uses FP16 or BF16 for computations (faster, less memory) while keeping FP32 for stability in some ops.
\end{itemize}

Full Example with FSDP + Checkpointing + AMP:

\begin{lstlisting}[language=Python]
import torch
import torch.nn as nn
import torch.optim as optim
import torch.distributed as dist
from torch.distributed.fsdp import FullyShardedDataParallel as FSDP
from torch.distributed.fsdp.wrap import transformer_auto_wrap_policy
from torch.utils.checkpoint import checkpoint

# --- A block with activation checkpointing ---
class CheckpointedBlock(nn.Module):
    def __init__(self):
        super().__init__()
        self.fc1 = nn.Linear(1024, 4096)
        self.act = nn.ReLU()
        self.fc2 = nn.Linear(4096, 1024)

    def forward(self, x):
        def forward_fn(x):
            return self.fc2(self.act(self.fc1(x)))
        # checkpoint will discard activations & recompute in backward
        return checkpoint(forward_fn, x)

# --- Toy Model ---
class ToyModel(nn.Module):
    def __init__(self, depth=6):
        super().__init__()
        self.layers = nn.Sequential(*[CheckpointedBlock() for _ in range(depth)])

    def forward(self, x):
        return self.layers(x)

def main():
    dist.init_process_group("nccl")
    torch.cuda.set_device(dist.get_rank() % torch.cuda.device_count())

    # Build model
    model = ToyModel().cuda()
    auto_wrap_policy = transformer_auto_wrap_policy
    model = FSDP(model, auto_wrap_policy=auto_wrap_policy)

    optimizer = optim.AdamW(model.parameters(), lr=1e-4)

    # Use mixed precision autocast (bf16 preferred if hardware supports it)
    scaler = torch.cuda.amp.GradScaler(enabled=True)  # works for fp16

    for step in range(20):
        x = torch.randn(8, 1024).cuda()

        with torch.cuda.amp.autocast(dtype=torch.bfloat16):  # or torch.float16
            y = model(x).mean()

        # backward with gradient scaler
        scaler.scale(y).backward()
        scaler.step(optimizer)
        scaler.update()
        optimizer.zero_grad()

        if dist.get_rank() == 0:
            print(f"Step {step} done.")

    dist.destroy_process_group()

if __name__ == "__main__":
    main()
\end{lstlisting}
\begin{itemize}
	\item Checkpointing: Wrap the forward function with \texttt{torch.utils.checkpoint.checkpoint}.
	\item AMP (Automatic Mixed Precision):
		\begin{itemize}
			\item Use \texttt{torch.cuda.amp.autocast} for forward pass.
			\item Use \texttt{torch.cuda.amp.GradScaler} for loss scaling (needed for FP16, not BF16).
		\end{itemize}
	\item BF16 vs FP16:
		\begin{itemize}
			\item Use BF16 if your GPUs are A100/H100 (more stable).
			\item Use FP16 + GradScaler for V100 or older cards.
		\end{itemize}
\end{itemize}


\begin{itemize}
	\item Full State Dict: Gather the full parameters on rank 0 and save them. (simpler, larger files).
	\item Sharded State Dict: Each rank saves only its shard. (efficient, but needs all shards to reload).
	\item Rank 0 only writing: Typically only rank 0 writes to disk to avoid file conflicts.
\end{itemize}
\begin{lstlisting}[language=Python]
import os
import torch
from torch.distributed.fsdp import FullyShardedDataParallel as FSDP
from torch.distributed.fsdp import StateDictType, FullStateDictConfig

CHECKPOINT_DIR = "./checkpoints"

def save_checkpoint(model, optimizer, step):
    # Ensure only rank 0 writes the file
    rank = torch.distributed.get_rank()
    os.makedirs(CHECKPOINT_DIR, exist_ok=True)

    # Switch to FULL state dict (gathered on rank 0)
	# offload_to_cpu=True: keeps memory usage down when gathering full state.
    full_sd_config = FullStateDictConfig(offload_to_cpu=True, rank0_only=True)
    with FSDP.state_dict_type(model, StateDictType.FULL_STATE_DICT, full_sd_config):
        model_state = model.state_dict()
        optim_state = optimizer.state_dict()

    if rank == 0: # only rank 0 writes to disk.
        save_path = os.path.join(CHECKPOINT_DIR, f"step_{step}.pt")
        torch.save({"model": model_state, "optimizer": optim_state, "step": step}, save_path)
        print(f"[Rank 0] Saved checkpoint to {save_path}")
\end{lstlisting}

\begin{lstlisting}[language=Python]
def load_checkpoint(model, optimizer, load_path):
    # Load only on rank 0
    rank = torch.distributed.get_rank()
    map_location = "cpu" if rank == 0 else "meta"  # meta avoids OOM on other ranks
    checkpoint = torch.load(load_path, map_location=map_location)

    # Use FULL state dict context
    full_sd_config = FullStateDictConfig(offload_to_cpu=True, rank0_only=True)
    with FSDP.state_dict_type(model, StateDictType.FULL_STATE_DICT, full_sd_config):
        model.load_state_dict(checkpoint["model"])

    # Broadcast model weights to all ranks
    torch.distributed.barrier()
    optimizer.load_state_dict(checkpoint["optimizer"])
    print(f"[Rank {rank}] Loaded checkpoint from {load_path}")
\end{lstlisting}

\begin{table}[h!]
\centering
\begin{tabular}{|p{2.5cm}|p{5cm}|p{6.5cm}|}
\hline
\textbf{Library} & \textbf{Core Idea} & \textbf{Strengths / Use Cases} \\
\hline
DeepSpeed & 
Distributed training engine with ZeRO (Zero Redundancy Optimizer) sharding & 
- ZeRO-1/2/3: memory savings via parameter/gradient/optimizer sharding \newline
- Offloading to CPU/NVMe \newline
- Mixture of Experts (MoE) support \newline
- Great for very large dense or MoE models \\
\hline
FSDP (Fully Sharded Data Parallel) & 
Native PyTorch module for full parameter, gradient, optimizer sharding & 
- First-party, stable, integrated in PyTorch \newline
- ZeRO-3–like memory scaling \newline
- Easy to use with Hugging Face Accelerate/Lightning \\
\hline
Megatron-LM & 
Parallelism library from NVIDIA (TP, PP, SP, EP) & 
- Tensor Parallelism (TP) for matmuls \newline
- Pipeline Parallelism (PP) for layer distribution \newline
- Sequence/Expert Parallelism for long-context and MoE \newline
- Standard for 30B–100B+ scale \\
\hline
\end{tabular}
\caption{Comparison of Core Libraries for Multi-GPU LLM Training}
\end{table}

\begin{itemize}
	\item Up to $~13$B on few GPUs $\to$ FSDP (simpler, first-party).
	\item >30B or MoE, multi-node $\to$ Megatron-LM + (FSDP or DeepSpeed ZeRO).
	\item When GPU memory is tight $\to$ DeepSpeed ZeRO-3 (with offload).
\end{itemize}

\section{Basic Training Techniques}

Unlike traditional ML models, LLMs are often trained in stages:

\begin{itemize}
	\item \href{https://optuna.org/}{Optuna}: Open source hyperparameter optimization (HPO) framework for machine learning, including Large Language Models (LLMs). 
\end{itemize}




