\documentclass[oneside, a4paper,11pt]{book}
\usepackage[margin=1in]{geometry}

\usepackage{graphicx}
\usepackage[utf8]{inputenc} % allow utf-8 input
\usepackage[T1]{fontenc}    % use 8-bit T1 fonts
% \usepackage{hyperref}       % hyperlinks
\usepackage[colorlinks=true, linkcolor=blue, urlcolor=blue, citecolor=green]{hyperref}
\usepackage{url}            % simple URL typesetting
\usepackage{booktabs}       % professional-quality tables
\usepackage{nicefrac}       % compact symbols for 1/2, etc.
\usepackage{microtype}      % microtypography

\usepackage{listings}
\usepackage{xcolor}
\definecolor{codegreen}{rgb}{0,0.6,0}
\definecolor{codegray}{rgb}{0.5,0.5,0.5}
\definecolor{codepurple}{rgb}{0.58,0,0.82}
\definecolor{backcolour}{rgb}{0.95,0.95,0.92}

\usepackage[most]{tcolorbox}

\newtcolorbox{commentbox}[1]{colback=blue!5!white,
  colframe=blue!75!black,
  fonttitle=\bfseries,
  title=#1}

\lstdefinestyle{mystyle}{
    backgroundcolor=\color{backcolour},
    commentstyle=\color{codegreen},
    keywordstyle=\color{magenta},
    numberstyle=\tiny\color{codegray},
    stringstyle=\color{codepurple},
    basicstyle=\ttfamily\footnotesize,
    breakatwhitespace=false,
    breaklines=true,
    captionpos=b,
    keepspaces=true,
    numbers=left,
    numbersep=5pt,
    showspaces=false,
    showstringspaces=false,
    showtabs=false,
    tabsize=2
}
\lstset{style=mystyle}


\usepackage[autostyle]{csquotes}
\usepackage{dsfont}

\usepackage{algorithm,algpseudocode}
% \usepackage[ruled,vlined]{algorithm2e}
\usepackage{multicol}
\usepackage{multirow}

\usepackage{hyperref}
\usepackage{amssymb}

\setlength{\parindent}{0pt}
\setlength{\parskip}{1em}

\newtheorem{theorem}{Theorem}
\newtheorem{definition}{Definition}
\newtheorem{proposition}{Proposition}
\newtheorem{corollary}{Corollary}

\newcommand\scalemath[2]{\scalebox{#1}{\mbox{\ensuremath{\displaystyle #2}}}}

\newcommand{\cyan}[1]{\textcolor{cyan}{#1}}

\input{math-com.tex}

\begin{document}

\begin{titlepage}
	\begin{center}
		\vspace*{5.5cm}
		\textbf{\Huge Deep Learning System Design}\\
        \vspace{2.5cm}
		\includegraphics[width=0.4\textwidth]{./logo/new_logo.pdf}\\
        % \Large Engineering and Service Architectures\\
        % % \Large My Note \\
        \vspace{1.0cm}
        % \vspace{1.5cm}
		Han Cheol Moon\\
		% School of Computer Science and Engineering\\
		% Nanyang Technological University\\
		% Singapore\\
		\texttt{tabularasa8931@gmail.com}
		\date{\today}
        \vspace{1.0cm}\\
		\small The logo depicts a cube puzzle gradually coming together, reflecting the journey of learning. Each piece represents a fragment of knowledge, and as they fall into place, they reveal the larger structure of understanding. It conveys the idea that growth is a process — knowledge is completed bit by bit.\\
	\end{center}
\end{titlepage}

% \frontmatter
% \maketitle
\tableofcontents
\newpage

\mainmatter
\part{Introduction}
\chapter{Introduction}

\section{Operations challenges with LLMs}
\begin{itemize}
	\item \textbf{Long download times} (\eg Bloom LLM is 330GB).
	\item \textbf{Longer deploy times} (\eg Bloom takes $30\sim 45$ mins to load the model into GPU).
	\item Along with increases in model size often come increases in \textbf{inference latency}. 
	\item \textbf{Managing GPUs}
	\item \textbf{Peculiarities of text data}: unlike other fields, texts have ambiguities. 
	\item \textbf{Token limits for a model} create bottlenecks
	\item \textbf{Hallucinations cause confusion} 
	\item \textbf{Bias and ethical considerations}
	\item \textbf{Security concerns}
	\item \textbf{Controlling costs}: \eg GPUs, infra, storage, operational costs like energy consumption during both training and inference. 
\end{itemize}

\section{LLMOps Essentials}

\begin{itemize}
	\item \textbf{Compression} is the practice of making models smaller. 
	\item \textbf{Quantizing} is the process of reducing precision in preference of lowering the memory requirements. 
\end{itemize}





\include{./sections/preliminaries}
\part{Data Engineering}
\chapter{Data Engineering for LLMs}

\textit{Data engineering} is the development, implementation, and maintenance of systems and processes that take in raw data and produce high-quality, consistent information that supports downstream use cases, such as analysis and machine learning. 

There isn't more valuable asset than your data. All successful AI and ML initiatives are built on a good data engineering foundation. It's important then that we acquire, clean, and curate our data. 
% Unlike other ML models, you generally won't be starting from scratch when creating an LLM customized for your specific task. 

\section{Models and the Foundation}
The most important dataset you will need to collect when training is the model weights of a pretrained model. 

\subsection{Evaluating LLMs}
When evaluating a model, you will need two things: \Ni a \textit{metric} and \Nii a \textit{dataset}. 

\paragraph{Metrics}
\begin{itemize}
	\item ROUGE (Recall-Oriented Understudy for Gisting Evaluation)
	\item BLEU (BiLingual Evaluation Understudy)
	\item BPC (\eg Perplexity): The bits per character (BPC) evaluation is an example of an entropy-based evaluation for language models. 
\end{itemize}

\paragraph{Industry benchmarks}
\begin{itemize}
	\item GLUE (General Language Understanding Evaluation) is essentially a standardized test for language models to measure performance versus humans and each other on language tasks meant to test understanding. 
	\item SuperGLUE
	\item MMLU (Massive Multitask Language Understanding). 
\end{itemize}

\paragraph{Responsible AI benchmarks}
\begin{itemize}
	\item HONEST evaluation metric compares how hurtful prompt completions are for different genders. 
	\item Some datasets: 
		\begin{itemize}
			\item WinoBias dataset focuses on gender bias. 
			\item CALM
			\item WinoQueer
		\end{itemize}
\end{itemize}

\paragraph{Developing your own benchmark}
\begin{itemize}
	\item \href{https://github.com/openai/evals}{OpenAI's Evals library} 
	\item \href{https://huggingface.co/docs/evaluate/en/index}{Huggingface's Evaluate}
\end{itemize}

\paragraph{Evaluating code generators}

The basic setup looks like this:
\begin{enumerate}
	\item Have your model generate code based on docstrings.
	\item Run the generated code in a safe environment on prebuilt tests to ensure they work and that no errors are thrown
	\item Run the generated code through a profiler and record the time it takes to complete. 
	\item Run the generated code through a security scanner and count the number of vulnerabilities. 
	\item Run the generated code against architectural fitness functions to determine artifacts like how much coupling, integrations, and internal dependencies there are. 
	\item Run steps 1 to 5 on another LLM.
	\item Compare results. 
\end{enumerate}

\paragraph{Evaluating model parameters}

There's a lot you can learn by simply looking at the parameters of an ML model. For instance, an untrained model will have a completely random distribution. 

\begin{lstlisting}[language=Python]
import weightwatcher as ww
from transformers import GPT2Model

gpt2_model = GPT2Model.from_pretrained("gpt2")
gpt2_model.eval()

watcher = ww.WeightWatcher(model=gpt2_model)
details = watcher.analyze(plot=False)
print(details.head())
#    layer_id       name         D  ...      warning        xmax        xmin
# 0         2  Embedding  0.076190  ... over-trained 3837.188332    0.003564
# 1         8     Conv1D  0.060738  ...              2002.124419  108.881419
# 2         9     Conv1D  0.037382  ...               712.127195   46.092445
# 3        14     Conv1D  0.042383  ...              1772.850274   95.358278
# 4        15     Conv1D  0.062197  ...               626.655218   23.727908

\end{lstlisting}
 
The spectral analysis plots evaluate the frequencies of eigenvalues for each layer of a model. These plots tell you whether a model (or layer) looks well-trained and generalizes well or is unstable/poorly conditioned.
Shape of the Spectrum (How eigenvalues are distributed)
\begin{itemize}
	\item Power-law exponent ($\alpha$):
		\begin{itemize}
			\item Good if between 2 and 6: the layer is well-trained.
			\item Bad if $\alpha > 6$: layer might be undertrained or over-regularized.
		\end{itemize}
	\item Fit quality (Dks):
		\begin{itemize}
			\item Low Dks: spectrum matches the expected ``heavy-tailed'' shape, reliable.
			\item High Dks: poor fit, unstable or unstructured layer.
		\end{itemize}
\end{itemize}

\section{Data for LLMs}
It has been shown that data is the most important part of training an LLM. 

% \begin{itemize}
% 	\item WikiText
% 	\item Wiki-40B
% 	\item Europarl
% 	\item Common Crawl
% 	\item OpenWebText
% 	\item The Pile
% 	\item RedPajama
% 	\item OSCAR
% \end{itemize}


\begin{table}[h]
	\setlength{\tabcolsep}{4pt}
	\caption{Summary of datasets}
	\centering
	\begin{tabular}{llrl}
		\toprule
		Dataset & Contents & Size & LastUpdate \\
		\midrule
		WikiText & English Wikipedia & $<$1GB & 2016 \\
		Wiki-40B & Multi-lingual Wikipedia & 10GB & 2020 \\
		Europarl & European Parliament proceedings & 1.5GB & 2011 \\
		Common Crawl & The internet & $\sim$ 300GB & Ongoing \\
		OpenWebText & Curated internet using Reddit & 55GB & 2019 \\
		The Pile & Everything above plus specialty datasets (books, law, med) & 825GB & 2020 \\
		\multirow{2}{*}{RedPajama} & GitHub, arXiv, Books, Wikipedia, StackExchange & 5TB & 2023 \\
								  &, and multiple version of Common Crawl&\\
		OSCAR & Highly curated multilingual dataset with 166 languages & 9.4TB & Ongoing\\
		\bottomrule
	\end{tabular}
\end{table}


\subsection{Data cleaning and preparation}
If you pulled any of the previously mentioned datasets, you might be surprised to realize most of them are just giant text dumps. There are no labels or annotations, and feature engineering hasn't been done at all. 

LLMs are trained via self-supervised manner to predict the next word or a masked workd, so a lot of traditional data cleaning and preparation processes are unneeded. This fact leads many to believe that data cleaning as a whole is unnecessary. 

Data cleaning and curation are difficult, time-consuming, and ultimately subjective tasks that are difficult to tie to key performance indicators (KPIs). Still, taking the time and resources to clean your data will create a more consistent and unparalleled user experience. 

The right frame of mind when preparing your dataset:
\begin{enumerate}
	\item Take your pie of data and determine a schema for the features
	\item Make sure all the features conform to a distribution that makes sense for the outcome your're trying to get through normalization or scaling.
	\item Check the data for bia/anomalies (most businesses skip this step by using automated checking instead of informed verification).
	\item Convert the data into a format for the model to ingest (for LLMs, it's through tokenization and embedding).
	\item Train, check, and retrain.
\end{enumerate}

\begin{commentbox}{Note}
	For more information, check out Fundamentals of Data Engineering, WizardLM, and LIMA: Less Is More for Alignment. 
\end{commentbox}


\paragraph{Instruct Schema} is one of the most effective and widely used data formats for fine-tuning models. Instruction tuning works on the principle that providing a model with explicit instructions for a task leads to better performance than simply giving it raw prompts and answers. In this approach, the data explicitly demonstrates what the model should do, making it clearer and more aligned with human intent. However, preparing such datasets is more demanding than assembling general web data, since each entry must be carefully constructed to match a structured format, typically including an instruction, optional input, and the expected output. You need to prepare your data to match a format that will look something like this:
\begin{lstlisting}[]
###Instruction

{user input}

###Input

{meta info about the instruction}

###Response

{model output}
\end{lstlisting}
It is a structured way of formatting data so that each example clearly contains:
\begin{itemize}
	\item An instruction (what the model should do).
	\item An input (optional context or data the model works on).
	\item An output (the desired response).
\end{itemize}

For instance, 
\begin{lstlisting}
{
  "instruction": "Translate the following English text into Korean.",
  "input": "The stock market saw significant volatility today due to global economic concerns.",
  "output": <Translations>
}
\end{lstlisting}

\begin{commentbox}{Note}
	\begin{itemize}
		\item EvolInstruct: WizardLM
		\item Self-instruct format, Alpaca
	\end{itemize}
\end{commentbox}

\paragraph{Ensuring proficiency with speech acts}
When preparing a dataset for training a model, the most important factor is ensuring the data truly reflects the task you want the model to perform. Misaligned or overly generic data reduces performance and can cause unpredictable behavior.

Dataset alignment:
\begin{itemize}
	\item Training data must match the intended task (e.g., don’t train on Titanic survivors if you want to predict Boston housing prices).
	\item In real-world use cases (like fast-food ordering), interactions are more diverse and unpredictable than generic datasets suggest.
\end{itemize}
Robustness and Risks:
\begin{itemize}
	\item Instruction datasets require intentional design: if a model is only trained on “helpful” responses, it might follow harmful instructions (e.g., “help me take over the world”).
	\item With tool access (Google, HR docs), this becomes even riskier.
\end{itemize}

Understanding speech acts (directives, representatives, commissives, expressives, declarations, verdictives) helps design datasets that match realistic user interactions.
\begin{itemize}
	\item In language learning, this means learners should not only know grammar/vocabulary but also how to perform speech acts appropriately:
		\begin{itemize}
			\item How to make polite requests
			\item How to refuse without sounding rude
			\item How to apologize or thank in culturally acceptable ways
		\end{itemize}
	\item In AI / LLM context, it means training the model to:
		\begin{itemize}
			\item Generate outputs that correctly perform the intended communicative function (e.g., distinguish between an instruction, a suggestion, or a formal declaration).
			\item Handle pragmatic nuances, politeness, indirectness, etc.
		\end{itemize}
\end{itemize}
Speech acts refer to the various functions language can perform in communication beyond conveying information. They are a way of categorizing utterances based on their intended effect or purpose in a conversation. In short, it is an action performed through speaking. For example: 
\begin{itemize}
	\item Assertives $\to$ stating something true/false.
	\item Directives $\to$ requesting, commanding (\eg ``Get it done in the next three days'').
	\item Commissives $\to$ promising, committing (\eg ``I swear'').
	\item Expressives $\to$ greetings, apologizing (\eg ``You are the best'').
	\item Declarations $\to$ enacting something by saying it (\eg ``I now pronounce you married'').
	\item Questions (\eg ``What is this?'')
\end{itemize}

\paragraph{Annotating the data}
Annotation is labeling your data, usually in a positionally aware way. For speech recognition tasks, annotations would identify the different words as noun, verb, adjective, or adverb. Annotations essentially give us metadata that makes it easier to reason about and analyze our datasets. 

There are tools to help with the task:
\begin{itemize}
	\item \href{https://prodi.gy/}{Prodigy}: multimodal annotation tool.
	\item \href{https://github.com/doccano/doccano}{doccano}: Open-source web-based platform for data annotation. 
	\item \href{https://www.fon.hum.uva.nl/praat/}{Praat}: The audio annotation tool.
	\item \href{https://galileo.ai/}{Galileo}: Galileo's LLM studio helps create prompt, evaluate and speed up annotation. 
\end{itemize}

\section{Text Processors}

We need to transform our dataset into something that can be consumed by the LLM. Simply, we need to turn the text into numbers.  

\subsection{Tokenization}
The tokenization is often ignored when working with an LLM through an API, but it is actually vitally important for every subsequent step, and it affects the LLM's performance significantly. 

\paragraph{Word-based}
Word-based tokenizers most commonly split on whitespace, but there are other methods like using regex, dictionaries, or punctuation. 

\paragraph{Character-based}
Character-based encoding methods are the most straightforward and easiest to implement since we split on the UTF-8 character encodings. However, it comes with a major loss of information and fails to keep relevant syntax, semantics, or morphology (morpheme like prefix and suffixes) of the text. 

\paragraph{Subword-based}
Subword-based tokenizers have proven to be the best option so far. 
\begin{itemize}
	\item BPE
	\item WordPiece
	\item SentencePiece
\end{itemize}

\subsection{Embeddings}
Embeddings provide meaning to the vectors generated during tokenization. 







\part{Training LLMs}
\chapter{Training LLMs: How to generate the generator}

\section{Multi-GPU environments}
Training is a resource-intensive endeavor. A model that only takes a single GPU to run inference on may take 10 times that many to train if, for nothing else, to parallelize your work and speed things up so you aren't waiting for a thousand years for it to finish training.  

\subsection{Setting up} 
It should be pointed out up front that while multi-GPU environments are powerful, they are also expensive. 

\part{Parallelism}
\chapter{Data Parallelism}

\begin{figure}[t]
	\centering
	\includegraphics[scale=1.5]{./images/data_parallel.pdf}
\end{figure}

\section{Data Parallel}
\label{sec:parallelism:data_parallelism:dp}

The first step of the typical training loop for deep learning models is to split a dataset into batches so that we can feed them into the model and compute gradients corresponding to them. As the model size grows up, we couldn't fit the model into a single GPU. The \textit{data parallelism} tries to tackle the issue by clone the model across multiple GPUs so that each GPU can take a small portion of the batches for each iteration. Data Parallel (sometimes referred to as ``single-node data parallel'') is typically used when you have \textbf{multiple GPUs on a single machine}. 

Let's say the batch size is 10 and we have 5 GPUs. Then, each GPU takes 2 batches and calculate gradients by on its own. The calculated gradients are then synchronized across the GPUs pretending they are computed on a single GPU. Finally, the synchronized gradient information is going to be distributed to them. 

There are some important things to mention: 
\begin{enumerate}
	\item One process (or master thread) becomes a bottleneck for gradient aggregation and parameter updates.
	\item As you increase the number of GPUs, or try to involve multiple machines, communication overhead grows significantly and can slow down training.
	\item Each GPU holds a copy of the entire model, which can be large.
\end{enumerate}

\section{Distributed Data Parallel}
\label{sec:parallelism:data_parallelism:ddp}
To alleviate such issues, we can adopt an approach called \textit{Distributed Data Parallel} (DDP), which is designed to scale training across many GPUs, potentially across multiple machines (nodes). Modern deep learning frameworks (like PyTorch torch.nn.parallel.DistributedDataParallel) typically recommend DDP as the best practice for multi-GPU/multi-node training due to better performance and scalability. During backpropagation, gradients are shared among GPUs through efficient communication primitives, resulting in synchronized model parameters across all GPUs.

Key benefits:
\begin{itemize}
	\item Scalability: You can increase the number of GPUs (and even add more machines) to handle large datasets and bigger models.
	\item Performance: DDP typically provides better performance than older methods like \textsc{nn.DataParallel} (in PyTorch) because it uses \textit{all-reduce} and eliminates the single ``master'' bottleneck.
	\item Flexibility: You can combine DDP with other parallelization strategies (\eg model parallel, sharded data parallel, pipeline parallel) if needed.
\end{itemize}

\subsection{Concepts and Terminology}

All-Reduce is a collective communication operation commonly used in distributed computing (especially in high-performance computing and deep learning). In simple terms:
\begin{itemize}
	\item Each process (or GPU) starts with its own data (\eg local gradients).
	\item These data are combined (usually via a reduction operation like sum, mean, min, or max) across all processes.
	\item The result of that reduction (\eg the summed gradients) is then shared back so that every process receives the same reduced value.
	\item Hence the name: ``all'' (everyone gets the result) + ``reduce'' (combine data).
\end{itemize}

Basic Terms:
\begin{itemize}
	\item World Size: The total number of processes engaged in the distributed job. Often, we run one process per GPU, so world size is the number of GPUs.
	\item Rank: A unique integer ID assigned to each process. Ranks typically range from 0 to $\text{world\_size} - 1$. Rank 0 is often referred to as the ``leader'' or ``master'' process, but in DDP, every process does roughly the same work.
	\item Local Rank: When multiple GPUs reside on a single node, local rank identifies which GPU a specific process is mapped to on that local machine (\eg 0 for the first GPU, 1 for the second, etc.).
	\item Backend: The communication backend used for synchronization (\eg nccl). For GPU training, NCCL is typically recommended because it's optimized for high-performance GPU-to-GPU communication.
	\item Initialization Method: Describes how processes connect with each other (\eg a TCP store, a file-based store). This allows all processes to know who's who in the cluster.
\end{itemize}

\subsection{How DDP Works Under the Hood}
\begin{enumerate}
	\item  Process Per GPU: Each GPU runs the same script in its own process.  
	\item  Data Subset: A DistributedSampler ensures that each process sees a unique subset of data. This prevents overlap in data usage among GPUs.  
	\item  Full Model Copy: Each GPU has a full replica of the model in memory.  
		\begin{itemize}
			\item For massive models, consider \textit{Sharded DDP} (\eg PyTorch's FSDP or DeepSpeed ZeRO) to split parameters across GPUs.
		\end{itemize}
	\item  All-Reduce Gradient Sync: After backprop, gradients are summed (or averaged) across processes with an all-reduce operation. This keeps all models in sync.
\end{enumerate}


\section{DDP vs DataParallel}

In PyTorch there are two common ways to use multiple GPUs:

\begin{itemize}
    \item \textbf{\texttt{nn.DataParallel}}: ``Single process, many GPUs, GPU0 is the boss.''
    \item \textbf{\texttt{DistributedDataParallel} (DDP)}: ``One process per GPU, talk via all-reduce (NCCL).''
\end{itemize}

In practice, \texttt{nn.DataParallel} is easy to use but inefficient and considered outdated for serious training. \texttt{DistributedDataParallel} is the recommended approach, especially for large models and LLMs.

\subsection{Mental Model}

\subsubsection{\texttt{nn.DataParallel}}

\begin{itemize}
    \item There is \textbf{one Python process} and one ``master'' model copy on GPU0.
    \item For each forward pass:
    \begin{enumerate}
        \item The input batch is split across GPUs.
        \item The model on GPU0 is \emph{replicated} onto the other GPUs.
        \item Forward and backward are run on each GPU.
        \item Gradients are gathered back to GPU0, summed, and \verb|optimizer.step()| is applied on GPU0's parameters.
    \end{enumerate}
    \item GPU0 becomes a bottleneck:
    \begin{itemize}
        \item It hosts the master model.
        \item It handles scatter/gather and the optimizer step.
    \end{itemize}
    \item Everything happens inside a single Python process, so the Python GIL can also limit scaling.
\end{itemize}

\subsubsection{\texttt{DistributedDataParallel} (DDP)}

\begin{itemize}
    \item There is \textbf{one process per GPU} (or per GPU per node).
    \item Each process:
    \begin{itemize}
        \item Owns a model replica on its GPU.
        \item Receives a shard of the global batch.
    \end{itemize}
    \item After \verb|loss.backward()|:
    \begin{itemize}
        \item Gradients are \textbf{averaged across processes} via an all-reduce operation (typically NCCL).
        \item Each process then calls \verb|optimizer.step()| locally using the synchronized gradients.
    \end{itemize}
    \item There is no single central bottleneck; synchronization is done in a distributed manner.
\end{itemize}

\subsection{Practical Differences}

\begin{center}
\begin{tabular}{|l|c|c|}
\hline
\textbf{Aspect} & \textbf{\texttt{nn.DataParallel}} & \textbf{\texttt{DistributedDataParallel}} \\
\hline
Processes & 1 & 1 per GPU \\
\hline
Model copies & Replicated each forward & One static replica per process \\
\hline
Gradient sync & Gather on GPU0 & All-reduce (NCCL) \\
\hline
Python GIL bottleneck & Yes & No (multi-process) \\
\hline
Multi-node support & No & Yes \\
\hline
Performance / scaling & Poor for large models / many GPUs & Good, recommended \\
\hline
Recommended for training & No & Yes \\
\hline
\end{tabular}
\end{center}

For large-scale ML and LLM training/fine-tuning, \texttt{DistributedDataParallel} is the default choice.


\chapter{Pipeline Parallelism}

\section{Introduction}
% \label{{sec:parallelism:pipeline_parallelism:intro}

The basic idea of the data parallel is to distribute the model across GPUs. However, if the model size is bigger than the VRAM of GPU, the model wouldn't fit in a single GPU. To resolve the issue, we have to split the model across GPUs. For instance, we can put the half of the model into the fist GPU and the remaining half into the second GPU. This approach is often called \textit{model parallelism}. Let's closely look at one of the model parallelism approaches, called \textit{pipeline parallelism}. 

\textbf{Pipeline Parallelism is a strategy for distributing large deep learning models across multiple devices (GPUs) by splitting the model layers into sequential stages.} Rather than replicating the entire model on each GPU or sharding the parameters themselves, pipeline parallelism assigns a subset of layers to each device in a pipeline-like fashion. This technique is especially helpful when:
\begin{itemize}
	\item The model is too large to fit on a single GPU, but it can be split into chunks (layers/stages).  
	\item You want to keep multiple GPUs actively working on different portions (stages) of the forward and backward pass concurrently.
\end{itemize}


\subsection{Illustration of the Pipeline}

In pipeline parallelism, the model is divided into $N$ sub-networks, and each sub-network is placed on a different GPU (or sometimes on multiple GPUs if you have many layers). Think of it like an assembly line:
\begin{itemize}
	\item Sub-Network 1: Layers $1\sim k$  
	\item Sub-Network 2: Layers $(k+1)\sim m$ 
	\item Sub-Network 3: Layers$(m+1)\sim \dots$ 
	\item and so on.
\end{itemize}

The input minibatch is then split into smaller micro-batches (smaller pieces of data), which flow sequentially through these sub-networks. In other words, the micro-batch is the basic unit of the input to the pipeline parallelism. 
\begin{itemize}
	\item While Stage 1 is processing the next micro-batch, Stage 2 can concurrently work on the intermediate outputs from Stage 1's previous micro-batch.
\end{itemize}

On each stage, for each microbatch that backpropagates through that stage's layers, the stage adds the microbatch gradients into its local grad buffers:

\begin{center}
	grads\_stage += dLoss/dParam (microbatch k)
\end{center}
Mixed precision usually applies loss scaling before accumulation.

\begin{figure}[t]
	\centering
	\includegraphics[scale=0.8]{./images/pipeline.pdf}
	\caption{The Illustration of the pipeline parallel on two GPUs. As you can see the \textit{bubble} (\ie underutilization) tends to grow as we increase the number of GPUs. It updates once at the end.}
\end{figure}

\paragraph{Example:} Imagine a 2-stage pipeline parallel setup (for simplicity):

\begin{itemize}
	\item GPU 0: Holds Layers 1–3  
	\item GPU 1: Holds Layers 4–6  
\end{itemize}

If you have a batch of data with 32 samples, you might split it into 4 micro-batches of size 8 each. Then, forward Pass can be processed as follows:
\begin{enumerate}
	\item Micro-Batch 1
		\begin{enumerate}
			\item Step A: GPU 0 processes layers 1–3 for micro-batch \#1.
			\item Step B: Once GPU 0 is done with those layers, it sends the activations for micro-batch \#1 over to GPU 1.
			\item Step C: GPU 1 then processes layers 4–6 for micro-batch \#1.
		\end{enumerate}
	\item Micro-Batch 2
		\begin{enumerate}
			\item As soon as GPU 0 finishes Step A for micro-batch \#1 and passes the data to GPU 1, GPU 0 is free to start micro-batch \#2 (layers 1–3).
			\item Meanwhile, GPU 1 is busy processing micro-batch \#1 (layers 4–6).
			\item Once GPU 0 finishes its part for micro-batch \#2, it sends those activations to GPU 1-which will be ready to handle them as soon as it's done with micro-batch \#1.
		\end{enumerate}
	\item Micro-Batch 3 and 4
		\begin{enumerate}
			\item This pattern continues in an overlapping fashion: while GPU 1 is busy with micro-batch \#2, GPU 0 can start on micro-batch \#3, and so on.
		\end{enumerate}
\end{enumerate}

The key benefit is concurrency:
\begin{itemize}
	\item While GPU 0 is processing micro-batch 2, GPU 1 can process micro-batch 1.  
	\item This overlap leads to higher GPU utilization.
\end{itemize}

Backward pass is a bit more complex because:
\begin{itemize}
	\item You need gradient signals to flow in the reverse order of the forward pipeline.  
	\item Each stage waits until it receives the gradient from the next stage before it can compute its own local gradients and pass them back to the previous stage.
\end{itemize}

However, the overall concept is similar-multiple stages can run backprop (on different micro-batches) in parallel, thereby keeping all GPUs busy.


\subsection{Pipeline Bubbles}

When using pipeline parallelism, you often hear about \textit{pipeline bubbles} (or underutilization). This refers to idle times on some GPUs before the assembly line is fully loaded or after it starts to wind down. 
\begin{itemize}
	\item Start-up Bubble: In the very beginning, GPU 1 must wait until GPU 0 finishes the first forward pass for micro-batch 1. GPU 1 sits idle during that initial delay.  
	\item Wind-down Bubble: After the last micro-batch enters GPU 0, GPU 1 continues to process the pipeline while GPU 0 is idle.
\end{itemize}

The percentage of idle can be computed as follows:
\begin{align*}
	\frac{1-m}{m+n-1},
\end{align*}
where $m$ is the number of microbatches and $n$ is the number of GPUs. 


These bubbles can lead to less-than-ideal speedups, but you can mitigate them by using enough micro-batches to keep the pipeline busy most of the time.

\subsection{Combining Pipeline Parallelism with Other Forms of Parallelism}

In practice, pipeline parallelism is often combined with:
\begin{itemize}
	\item Data Parallelism: You still replicate each stage across multiple GPUs to handle separate shards of data.  
	\item Tensor Parallelism / Model Parallelism: Instead of giving entire layers to one GPU, you split the parameters or compute of a single layer across multiple GPUs (common in large language model setups, \eg Megatron-LM).  
	\item Sharded Optimizer Approaches (\eg ZeRO, FSDP): Distribute optimizer states and gradients to reduce memory overhead.
\end{itemize}


\section{1F1B}

\begin{figure}[t]
	\centering
	\includegraphics[scale=0.23]{./images/1f1b.png}
\end{figure}
One of the issues is that the model parameters keep changing while processing the forward passes. This means at every time step, minibatches are going to be forwarded through different weights. Thus, it is necessary to keep different states of the model parameters. Thus, 1F1B increases the memory requirements while increasing the processing speed. 

\subsection{Non-interleaved Schedule}

The non-interleaved schedule can be divided into two states. The first state is the startup state (or warm-up state). In the startup state, After completing the forward pass for the first minibatch, it performs the backward pass for the same minibatch, and then starts alternating between performing forward and backward passes for subsequent minibatches. As the backward pass starts propagating to earlier stages in the pipeline, every stage starts alternating between forward and backward pass for different minibatches. As shown in the above figure, in the steady state, every machine is busy either doing the forward pass or backward pass for a minibatch.

\subsection{Interleaved Schedule}

This schedule requires the number of microbatches to be an integer multiple of the stage of pipeline. In this schedule, each device can perform computation for multiple subsets of layers(called a model chunk) instead of a single contiguous set of layers. \ie Before device 1 had layer 1-4; device 2 had layer 5-8; and so on. But now device 1 has layer 1,2,9,10; device 2 has layer 3,4,11,12; and so on. With this scheme, each device in the pipeline is assigned multiple pipeline stages and each pipeline stage has less computation. This mode is both memory-efficient and time-efficient.



\section{Zero Bubble}
\label{sec:}

\chapter{Tensor Parallelism}

\section{Introduction}

\begin{figure}[t]
	\centering
	\includegraphics[scale=0.8]{./images/tensor_parallel.pdf}
	\caption{(a): Pipeline parallelism. (b) Tensor parallelism.}
\end{figure}

Let's go over an example:
\begin{itemize}
	\item \( x \) is a row vector of shape \([1, d_\text{in}]\) (the input).  
	\item \( W \) is a weight matrix of shape \([d_\text{in}, d_\text{out}]\).  
	\item \(\text{output}\) is \([1, d_\text{out}]\).
\end{itemize}

We have two GPUs, GPU 0 and GPU 1. We want to split (shard) the weight matrix \( W \) across two GPUs. One common approach is column parallelism:  
\begin{itemize}
	\item GPU 0 holds columns \([0,1]\)  
	\item GPU 1 holds columns \([2,3]\)  
\end{itemize}

This means each GPU stores some columns of \(W\). Let's denote:

\[
W = \bigl[W_{\text{left}} \,\big|\ W_{\text{right}}\bigr]
\]

where
\begin{itemize}
	\item \( W_{\text{left}} \) is a \(4 \times 2\) matrix on GPU 0,  
	\item \( W_{\text{right}} \) is a \(4 \times 2\) matrix on GPU 1.
\end{itemize}

In numeric form, suppose
\[
W =
\begin{bmatrix}
1 & 2 & 5 & 6\\
3 & 4 & 7 & 8\\
2 & 0 & 3 & 1\\
-1 & 4 & 8 & 2
\end{bmatrix}.
\]

Then, for column parallel:
\begin{itemize}
	\item GPU 0:  
  \[
  W_{\text{left}} = 
  \begin{bmatrix}
  1 & 2 \\
  3 & 4 \\
  2 & 0 \\
  -1 & 4 
  \end{bmatrix}.
  \]
\item GPU 1:  
  \[
  W_{\text{right}} = 
  \begin{bmatrix}
  5 & 6 \\
  7 & 8 \\
  3 & 1 \\
  8 & 2
  \end{bmatrix}.
  \]
\end{itemize}

Given the input
\[
x = [\, 1,\ 2,\ 0,\ 1\,].
\]

We can treat \(x\) as a row vector \([1,4]\). For column parallelism, each GPU needs the entire input \(x\) so it can multiply by its subset of columns:
\begin{itemize}
	\item We copy the \( x \) to both GPU 0 and GPU 1.  
		\begin{itemize}
			\item This is typically a small overhead compared to storing large weight matrices.
		\end{itemize}
	\item Then, compute the matrix multiplications for each matrix.
	\item Finally, concatenate the outputs.
\end{itemize}

\[
\text{output} = [\, \text{partial}_0 \;\big|\; \text{partial}_1\,] = [\,6,\ 14,\ 27,\ 24\,].
\]
\begin{itemize}
	\item Some frameworks do a ring-all-gather, or they might place this final output on one GPU if needed, etc. 
\end{itemize}

When we do backprop, we can update the model's parameters in the opposite direction. 


In Megatron-LM, all Transformer layers, except normalization layer, are using row or column parallelism. 


Tensor parallelism can be costly primarily due to the significant communication overhead involved when distributing large model layers across multiple GPUs, requiring frequent data exchange between devices which can become a bottleneck, especially when dealing with very large models and limited network bandwidth; this communication cost often outweighs the benefits of parallel computation, making it a major drawback of tensor parallelism. 

\include{./sections/parallelism/n_dim_parallel}
\chapter{DualPipe}


\section{All-to-All vs Point-to-Point}
% \label{sec:}

When orchestrating multiple GPUs, we need them to communicate with each other to share information like gradients and model parameters. There are two main types of communication patterns: 
\begin{enumerate}
	\item All-to-all communication.
	\item Point-to-point communication.
\end{enumerate}

\textit{All-to-all communication} involves every GPU in the system simultaneously exchanging data with all other GPUs. The canonical analogy is a group chat where everyone needs to share their updates with everyone else. All-to-all communication is expensive and involves a ton of communication overhead. There are several clever algorithms like ring-AllReduce that can reduce this overhead, but it's still often a bottleneck.

\textit{Point-to-point communication}, on the other hand, is a communication between just two GPUs (the analogy here is a private conversation). One GPU sends data directly to another specific GPU without involving the rest of the system. This is much more efficient in terms of network bandwidth and latency. In practice, point-to-point communication is strongly preferred when possible because it's significantly cheaper in terms of computational resources.




\part{Transformers}

\section{Flash Attention}
\label{sec:transformer:flash_attention}

\chapter{Positional Embeddings}
\label{ch:transformer:posemb}

Rather than focusing on a token's absolute position in a sentence, relative positional embeddings concentrate on the distances between pairs of tokens. This method doesn't add a position vector to the word vector directly. Instead, it alters the attention mechanism to incorporate relative positional information.

\subsection{Rotary Positional Embeddings}

\chapter{Tokenization}
\label{ch:transformer:tokenization}

\chapter{Model Compression}
\label{ch:transformer:compression}



\part{Compression}
\chapter{Model Compression}
\chapter{Introduction}

\section{Operations challenges with LLMs}
\begin{itemize}
	\item \textbf{Long download times} (\eg Bloom LLM is 330GB).
	\item \textbf{Longer deploy times} (\eg Bloom takes $30\sim 45$ mins to load the model into GPU).
	\item Along with increases in model size often come increases in \textbf{inference latency}. 
	\item \textbf{Managing GPUs}
	\item \textbf{Peculiarities of text data}: unlike other fields, texts have ambiguities. 
	\item \textbf{Token limits for a model} create bottlenecks
	\item \textbf{Hallucinations cause confusion} 
	\item \textbf{Bias and ethical considerations}
	\item \textbf{Security concerns}
	\item \textbf{Controlling costs}: \eg GPUs, infra, storage, operational costs like energy consumption during both training and inference. 
\end{itemize}

\section{LLMOps Essentials}

\begin{itemize}
	\item \textbf{Compression} is the practice of making models smaller. 
	\item \textbf{Quantizing} is the process of reducing precision in preference of lowering the memory requirements. 
\end{itemize}







\backmatter
% bibliography, glossary and index would go here.

\nocite{*}
\bibliographystyle{unsrt}
\bibliography{references}
\end{document}
